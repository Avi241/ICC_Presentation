%%%%%%%%%%%%%%%%%%%%%%%%%%%%%%%%%%%%%%%%%
% Beamer Presentation
% LaTeX Template
% Version 1.0 (10/11/12)
%
% This template has been downloaded from:
% http://www.LaTeXTemplates.com
%
% License:
% CC BY-NC-SA 3.0 (http://creativecommons.org/licenses/by-nc-sa/3.0/)
%
%%%%%%%%%%%%%%%%%%%%%%%%%%%%%%%%%%%%%%%%%

%----------------------------------------------------------------------------------------
%	PACKAGES AND THEMES
%----------------------------------------------------------------------------------------

\documentclass[10pt]{beamer}

\mode<presentation> {

% The Beamer class comes with a number of default slide themes
% which change the colors and layouts of slides. Below this is a list
% of all the themes, uncomment each in turn to see what they look like.

%\usetheme{default}
%\usetheme{AnnArbor}
%\usetheme{Antibes}
%\usetheme{Bergen}
%\usetheme{Berkeley}
%\usetheme{Berlin}
%\usetheme{Boadilla}
\usetheme{CambridgeUS}
%\usetheme{Copenhagen}
%\usetheme{Darmstadt}
%\usetheme{Dresden}
%\usetheme{Frankfurt}
%\usetheme{Goettingen}
%\usetheme{Hannover}
%\usetheme{Ilmenau}
%\usetheme{JuanLesPins}
%\usetheme{Luebeck}
%\usetheme{Madrid}
%\usetheme{Malmoe}
%\usetheme{Marburg}
%\usetheme{Montpellier}
%\usetheme{PaloAlto}
%\usetheme{Pittsburgh}
%\usetheme{Rochester}
%\usetheme{Singapore}
%\usetheme{Szeged}
%\usetheme{Warsaw}

% As well as themes, the Beamer class has a number of color themes
% for any slide theme. Uncomment each of these in turn to see how it
% changes the colors of your current slide theme.

%\usecolortheme{albatross}
%\usecolortheme{beaver}
%\usecolortheme{beetle}
%\usecolortheme{crane}
\usecolortheme{dolphin}
%\usecolortheme{dove}
%\usecolortheme{fly}
%\usecolortheme{lily}
%\usecolortheme{orchid}
% \usecolortheme{rose}
%\usecolortheme{seagull}
%\usecolortheme{seahorse}
%\usecolortheme{whale}
%\usecolortheme{wolverine}


%\setbeamertemplate{footline} % To remove the footer line in all slides uncomment this line
\setbeamertemplate{footline}[frame number] % To replace the footer line in all slides with a simple slide count uncomment this line
\setbeamertemplate{navigation symbols}{} % To remove the navigation symbols from the bottom of all slides uncomment this line
}
\usepackage{graphicx} % Allows including images
\usepackage{subcaption}
\usepackage{booktabs} % Allows the use of \toprule, \midrule and \bottomrule in tables
\usepackage{wrapfig}
\usepackage{multirow}
\usepackage{lipsum}
\usepackage{xcolor}
\usepackage{amsmath,amssymb}
\usepackage{array}
\usepackage{gensymb}
\usepackage{animate,media9,movie15}
%\usepackage{parskip}
\usepackage[english]{babel}
\usepackage{tikz}
%\usetikzlibrary{shapes,arrows}
%\usetikzlibrary{positioning,fit,calc}
\usefonttheme{serif}
\usetikzlibrary{arrows.meta}
\usepackage{ragged2e}
\usepackage{tikz}
\tikzset{>=latex}
\usepackage{amssymb,amsfonts,amsmath}
\usepackage{tikz,tkz-euclide}
\usepackage{appendixnumberbeamer}
\newtheorem{proposition}{Proposition}

%----------------------------------------------------------------------------------------
%	TITLE PAGE
%----------------------------------------------------------------------------------------

\title[]{ROS-Based \textbf{M}ulti-\textbf{A}gent \textbf{S}ystems \textbf{CO}ntrol Simulation \textbf{T}estbed (MASCOT)} % The short title appears at the bottom of every slide, the full title is only on the title page

\author[]{Arvind Pandit, Akash Njattuvetty, and Ameer K. Mulla
} % Your name
\titlegraphic{
	\includegraphics[origin=c,scale=0.25]{icc_l.png}\qquad\qquad\includegraphics[scale=0.04]{IITDh_Logo.png}
}

\institute[] % Your institution as it will appear on the bottom of every slide, may be shorthand to save space
{
	Department of Electrical Engineering\\
	Indian Institute of Technology Dharwad, India
	%\medskip
}
\date{Indian Control Conference (ICC-8) \\ 14-16 December 2022, Chennai, India.}

\begin{document}

	
\begin{frame}[plain,noframenumbering]
	%[noframenumbering]
	\titlepage % Print the title page as the first slide
%	\begin{figure}
%		\centering \includegraphics[scale=cale=0.5]{IITDh_Emblem.jpg}
%	\end{figure}
\end{frame}

\begin{frame}
\frametitle{Overview} % Table of contents slide, comment this block out to remove it
\tableofcontents % Throughout your presentation, if you choose to use \section{} and \subsection{} commands, these will automatically be printed on this slide as an overview of your presentation
\end{frame}

%----------------------------------------------------------------------------------------
%	PRESENTATION SLIDES
%----------------------------------------------------------------------------------------

%----------------------------------------------------------
\section{Introduction}
\subsection{Multi-Agent Systems}

%----------------------------------------------------------
\begin{frame}
\frametitle{Multi-Agent Systems}

\end{frame}
%----------------------------------------------------------.
\subsection{Simulation Platform for Multiagent System}

\begin{frame}
\frametitle{Simulation Platform for Multiagent System}

\end{frame}

\subsection{MASCOT}

\begin{frame}
\frametitle{MASCOT}

\end{frame}
%------------------------------------------------------------.
\section{Preliminaries}
%-----------------------------------------------------------
\begin{frame}
\frametitle{Preliminaries}
\end{frame}
%------------------------------------------------------------
\subsection{Frame of Reference}
%-----------------------------------------------------------/
\begin{frame}
\frametitle{Frame of Reference}
\end{frame}
%%-----------------------------------------------------------
%%-----------------------------------------------------------
\subsection{Quadcopter Dynamics}
%------------------------------------------------------------
\begin{frame}
\frametitle{Quadcopter Dynamics}
\end{frame}
%%--------------------------------.
%%----------------------------------------------------------
\subsection{Quadcopter Dynamics as Double Integrator}
%-----------------------------------------------------------
\begin{frame}
\frametitle{Quadcopter Dynamics as Double Integrator}
\end{frame}
%%---------------------------------------------------------
\section{MASCOT:Structure and Features}

\begin{frame}
\frametitle{MASCOT:Structure and Features}
\end{frame}

\subsection{Tools Used}
%----------------------------------------------------------
\begin{frame}
\frametitle{Tools Used}
\end{frame}
%--------------------------------------------------------------
\subsection{Control Block}
\begin{frame}
\frametitle{Control Block}
\end{frame}

\subsection{Architecture}
\begin{frame}
\frametitle{Architecture}
\end{frame}

\subsection{Features}
\begin{frame}
\frametitle{Features}
\end{frame}

\subsection{Configuration}
\begin{frame}
\frametitle{Configuration}
\end{frame}

%-----------------------------------------------------
\section{Examples}
%-----------------------------------------------------
\begin{frame}
\frametitle{Examples}
\end{frame}

\subsection{Way-Point Navigation}
\begin{frame}
\frametitle{Way-Point Navigation}
\end{frame}

\subsection{Consensus Algorithm}
\begin{frame}
\frametitle{Consensus Algorithm}
\end{frame}

\subsection{Min-Max Time Consensus}
\begin{frame}
\frametitle{Min-Max Time Consensus}
\end{frame}


\section{Consclusions and Future Work}
%-----------------------------------------------------
\begin{frame}
\frametitle{Consclusions and Future Work}
\end{frame}

%----------------------------------------------------------------------------------------

\end{document}

